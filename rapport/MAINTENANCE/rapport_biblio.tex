\chapter{Bibliographie}


\begin{itemize}

\item[$\bullet$] {\bf {\underline http://psidev.sourceforge.net/ }}
Le site de PSI fournit toutes les informations sur la {\bf P}roteomics
{\bf S}tandards {\bf I}nitiative, plus particuli\`erement sur la XSD de 
PSI. Il est possible d'avoir acc\`es aux mailing lists ce qui nous a 
permis d'\^etre au courant des modifications de la norme PSI. \\
\item[$\bullet$] {\bf {\underline http://dip.doe-mbi.ucla.edu/ }}
Le site de DIP nous a fourni les exemples qui ont \'et\'e utilis\'es 
pour tester nos outils, ainsi que la norme DIP.\\
\item[$\bullet$] {\bf {\underline http://www.w3.org/Style/XSL/ }}
Ce site donne la sp\'ecification de XSL, les \'el\'ements que l'on peut 
utiliser et leur fonctionnalit\'es.\\
\item[$\bullet$] {\bf {\underline http://www.w3.org/XML/Schema }}
Ce site donne la d\'efinition des schemas XML.\\
\item[$\bullet$] {\bf {\underline http://www.enseirb.fr/~thomas-n/PFA/}}
Ce site a \'et\'e r\'ealis\'e par un membre de notre \'equipe pour 
permettre une communication plus efficace. Il 
contient un forum de discussion o\`u \'etaient d\'epos\'ees les remarques, 
les compte-rendus de r\'eunions avec le client, ainsi que la r\'epartition 
des t\^aches associ\'ees \`a chaque \'etape du projet.\\

\end{itemize}