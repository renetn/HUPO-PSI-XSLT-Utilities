\chapter{Glossaire}

\begin{itemize}

\item[$\bullet$]{\bf CSV} : Coma Separated Value, ou CSV, est un format de donn�es. Un fichier de ce type contient des valeurs dans une table repr�sent�e par une s�rie de lignes de caract�res organis�e de telle mani�re que chaque colonne est s�par�e de la colonne suivante par un point-virgule (ou une virgule, ou une tabulation...) et que chaque retour chariot d�signe la fin d'une ligne.\\

\item[$\bullet$] {\bf DIP} : Database of Interacting Proteins, ou DIP d�signe une base de donn�es d'interactions prot�ine-prot�ine.\\

\item[$\bullet$]{\bf Processeur XSLT} : Le processeur XSLT, composant logiciel charg� de la transformation du fichier XML, cr�e un structure logique arborescente � partir du document XML et lui fait subir des transformations selon les \texttt{templates} contenus dans la feuille XSL pour produire un arbre r�sultat repr�sentant la structure d'un nouveau document au format XML, CSV, HTML...\\ 
\item[$\bullet$] {\bf PSI} : Proteomic Standard Initiative, ou PSI, d�signe un format de donn�es. Ce format a �t� d�velopp� afin de standardiser les format de donn�es relatives aux interactions prot�ine-prot�ine. Il existe deux variantes de ce format nomm�es PSI canonique et PSI d\'eroul\'e.\\

\item[$\bullet$]{\bf Template} : Chaque \texttt{template} d�finit des traitements � effectuer sur un �l�ment (noeud ou feuille) de l'arborescence XML.\\

\item[$\bullet$]{\bf XIN} : Le format XIN est un format XML tr�s flexible qui permet de d�crire
des graphes. La base de donn�e DIP est une application particuli�re de
ce format.\\

\item[$\bullet$]{\bf XSD} : XML Schema Definition. Fichier XML donnant la d�finition d'un format de fichier XML\\

\item[$\bullet$]{\bf XML} : eXtensible Markup Language, ou XML, est un m�ta-langage extensible d�riv� de SGML permettant de structurer des donn�es.\\

\item[$\bullet$]{\bf XSLT/feuille de style} : eXtensible Stylesheet Language Transformation, ou XSLT, est un langage permettant de transformer la structure des fichiers au format XML. Ainsi, un document XML peut �tre repr�sent� comme une structure arborescente. XSLT permet alors de transformer les documents XML � l'aide de feuilles de style contenant des r�gles appel�es \texttt{template}.\\






\end{itemize}