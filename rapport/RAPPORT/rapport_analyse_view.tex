\section{Description de l'outil de visualisation PSI}
Il sagit d'un outil qui � partir d'un fichier PSI canonique retourne un fichier au format HTML qui pr�sente les int�ractions de fa�on synth�tique.\\
La phase d'analyse de cet outil � �t� faites par Henning Hermjakob (acteur du projet PSI) qui nous a donn� une version bien avanc�e de cet outil. Notre travail a consist� reprendre l'impl�mentation et � refaire la mise en forme de se document de fa�on � pr�senter le mieux possible les r�sultats.

\section{Contenu du fichier}
\begin{itemize}
\item[$\bullet$] On affiche d'abord les sources et les licences du fichiers.
\item[$\bullet$] On construit un tableau dont chaque ligne est un participant � une int�raction. Le tableau pr�sente les donn�es : {\em interactorRef/@ref} or {\em proteinInteractor/names/shortLabel}, {\em participant/role}, {\em participant/isTaggedProtein}, {\em participant/isOverexpressedProtein}, {\em interactionType}.
\item[$\bullet$] Pour plus de lisibilit� les lignes sont regroup�es par int�raction et on alterne la couleur d'affichage d'un partiant � chaque linge.
\end{itemize}


