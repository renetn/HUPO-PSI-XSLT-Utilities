\section{Description de l'outil de visualisation PSI}


\section{Contenu du fichier}

\subsection{Sources}
Dans cette partie du fichier on affiche le noms des sources du fichier � partir du champ {\em source/names/shortLabel}.

\subsection{Availability List}
Dans cette partie du fichier on affiche les licences \`a partir du fichier � partir du champ {\em avaibility}.

\subsection{Interaction List}
On construit un tableau dont chaque ligne est un participant � une int�raction. Le tableau pr�sente les donn�es : {\em interactorRef/@ref} or {\em proteinInteractor/names/shortLabel}, {\em participant/role}, {\em participant/isTaggedProtein}, {\em participant/isOverexpressedProtein}, {\em interactionType}.\\
\\
Pour plus de lisibilit� les lignes sont regroup�es par int�raction et on alterne la couleur d'affichage d'un partiant � chaque linge.




