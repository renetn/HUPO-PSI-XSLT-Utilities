\chapter{Installation et proc�dure d'utilisation}

\section{Proc�dure d'utilisation}
Le format PSI est un format XML particulier. Par ailleurs, les outils pr�sent�s ici sont des feuilles de style au format XSLT. Par cons�quent, pour utiliser cet outil, il nous faut appliquer la feuille de style au fichier PSI.\\

Pour cela, il faut faire appel � un processeur XSLT. Il s'agit en fait d'un programme permettant d'appliquer une feuille XSLT � un fichier XML. Il existe de nombreux processeurs r�alisant cette transformation dont la syntaxe d'utilisation varie d'un processeur � l'autre. Nous allons ici pr�senter celui que nous avons utilis� lors de la r�alisation de nos outils. Les moyens disponibles pour obtenir ce processeur et pour l'installer sont d�crits un peu plus loin dans la section {\it Installation}.\\

En ce qui concerne \texttt{xsltproc}, pour appliquer notre outil (nomm� ici {\it outil.xsl}) � un fichier PSI (nomm� ici {\it fichier\_PSI.xml}) et pour stocker le r�sultat dans un fichier nomm� {\it resultat}, il suffit de taper la ligne de commande suivante :

\begin{center}
\verb+ xsltproc -o resultat outil.xsl fichier_PSI.xml+
\end{center} 

Le fichier {\it resultat} a un format qui va varier d'un outil � l'autre. Nous invitons donc l'utilisateur � se reporter � la section {\it Usage} de chaque outil pour connaitre le format exact du fichier en sortie ainsi que les noms pr�cis des feuilles de style � appliquer (nomm�es ici {\it outil.xsl}).
\newpage
\section{Installation}
Comme nous l'avons expliqu� dans la section pr�c�dente, le seul logiciel n�cessaire pour appliquer notre outil � un fichier PSI est un processeur XSL. Par exemple, vous pouvez trouver une version t�l�chargeable du processeur \texttt{xsltproc} � l'adresse internet suivante :
\begin{center}
\underline{www.xmlsoft.org}
\end{center} 

Le processeur se trouve en fait dans la librairie \texttt{libxml} disponible sur le site de xmlsoft. Il suffit donc d'installer cette librairie pour avoir acc�s au processeur \texttt{xsltproc}.
