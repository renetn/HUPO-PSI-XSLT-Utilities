\chapter{Introduction}

Actuellement, l'ensemble des informations en prot�omique, au niveau international, sont stock�es dans des formats de donn�es vari�s. Notamment, on trouve des donn�es au format XML (DIP,...) et au format CSV. Malheureusement, la multitude de ces formats rend contraignants les �changes et comparaisons de donn�es. La Proteomics Standards Initiative (PSI) a pour but de d�finir un format standard pour la repr�sentation de donn�es en prot�omique afin de faciliter la comparaison, l'�change et la v�rification d'informations.\\

Le but de ce manuel est de pr�senter un ensemble d'outils permettant de faire des transformations de format existants vers PSI et inversement. Ainsi, apr�s une description des proc�dures menant � la bonne utilisation des outils, nous nous attarderons ici sur les outils eux-m�mes.

Plus pr�cs�ment, nous pr�senterons ici les outils suivants :\\
\begin{itemize}

\item[$\bullet$] Le format PSI poss�dant deux ``variantes'' (PSI canonique et PSI d\'eroul\'e), nous pr�senterons d'abord deux outils permettant le passage d'un format � l'autre dans les deux sens.\\

\item[$\bullet$] Relativement au format PSI, nous d�crirons �galement ici deux outils de v�rification : l'un sert � certifier qu'un fichier est au format PSI canonique et l'autre effectue la m�me v�rification pour un fichier de type PSI d\'eroul\'e.\\

\item[$\bullet$] Nous pr�senterons ensuite un outil permettant de passer du format DIP au format PSI canonique ainsi qu'un test de v�rification de l'int�grit� du fichier PSI en sortie.\\

\item[$\bullet$] Enfin, nous d�crirons un outil permettant de transformer un fichier au format PSI canonique en un fichier au format CSV.\\

Pour aider � la compr�hension de ce document, le lecteur pourra trouver un petit glossaire � la fin du manuel.

\end{itemize}
 



