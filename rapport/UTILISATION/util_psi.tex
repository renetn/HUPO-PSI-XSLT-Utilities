\chapter{Outils concernant les transformations entre formats PSI}
Il existe quatre outils concernant ces transformations :
\begin{itemize}
\item MIF\verb+_+unfold.xsl qui permet de passer du format PSI canonique \`a PSI d\'eroul\'e.
\item MIF\verb+_+fold.xsl qui permet de passer du format PSI d\'eroul\'e \`a PSI canonique.
\item Et deux fichiers de tests :
\begin{itemize}
\item MIF\_normalisedFormTester.xsl qui permet de v\'erifier qu'un fichier est bien en PSI canonique
\item MIF\_denormalisedFormTester.xsl qui permet de v\'erifier qu'un fichier est bien en PSI d\'eroul\'e
\end{itemize}
\end{itemize}

\section{MIF\_unfold.xsl}
\subsection{Pr\'esentation}
Cet outil permet \`a partir d'un fichier XML PSI sous forme canonique d'obtenir un fichier au format XML PSI sous forme d\'eroul\'e.
\subsection{Fonctionnalit\'e}
En entr\'ee il faut un fichier XML dans le format PSI canonique bien form\'e et conforme � la XSD. Il existe l'outil MIF\_normalisedFormTester.xsl qui permettra si on le souhaite de tester que le fichier d'entr\'ee est bien form\'e (cf partie suivante).\\
\\
En sortie, nous obtenons un fichier XML dans le format PSI d\'eroul\'e. Ce fichier contient les m\^emes informations que celles contenues dans le fichier de d\'epart si celui-ci \'etait bien form\'e.

\subsection{Usage}
Pour mettre en oeuvre cette transformation, il faut utiliser la ligne de commande suivante :
\begin{center}
xsltproc -o \verb+<+nom du fichier d'arriv\'ee.xml\verb+>+ MIF\verb+_+unfold.xsl \verb+<+nom du fichier d'entr\'ee\verb+>+
\end{center}
(cf chapitre xsltproc pour plus d'informations)

\subsection{D\'epannage}
Si les conditions de d\'epart ne sont pas v\'erifi\'ees, c'est \`a dire que le fichier d'entr\'ee n'est pas bien form\'e, plusieurs cas de figure se pr\'esentent:
\begin{itemize}
\item Il peut y avoir des r\'ef\'erences \`a rien. Dans ce cas, le r\'esultat sera totalement erron\'e.
\item Il peut y avoir des descriptions dans les int\'eractions au lieu de r\'ef\'erences. Dans ce cas, le r\'esultat reste correct.
\end{itemize}

\section{MIF\_normalisedFormTester.xsl}
\subsection{Pr\'esentation}
Ce fichier est un test permettant de v\'erifier qu'un fichier PSI est bien PSI canonique.

\subsection{Fonctionnalit\'e}
En entr\'ee, il faut un fichier XML de format PSI.

En sortie, on a fichier rapport au format HTML qui renvoie la liste des erreurs (s'il y en a) que l'on a rencontr\'e et qui font que le fichier d'entr\'ee n'est pas en PSI canonique. Ces erreurs peuvent \^etre les suivantes :
\begin{itemize}
\item Celles appel\'ees {\it  List of misplaced descriptions :} lorsque l'on a une description dans les int\'eractions au lieu d'une r\'ef\'erence. \\Si ce champ est vide c'est qu'il n'y a aucune erreur. Par contre quand il y a une erreur, plusieurs informations sont indiqu\'ees :
\begin{itemize}
\item Le num\'ero de l'int\'eraction dans laquelle l'erreur a \'et\'e d\'etect\'ee.
\item L'attribut {\it id} de la description concern\'ee.
\end{itemize}
\item Celles appel\'ees {\it  List of undefined references :} qui indiquent s'il existe des r\'ef\'erences vers des descriptions qui n'existent pas.\\Si ce champ est vide, c'est qu'il n'y a aucune erreur. Par contre, lorsqu'il y a une erreur, plusieurs informations sont indiqu\'ees :
\begin{itemize}
\item Le num\'ero de l'int\'eraction dans laquelle l'erreur a \'et\'e d\'etect\'ee.
\item L'attribut {\it ref} de la r\'ef\'erence concern\'ee.
\end{itemize}
\end{itemize}
\subsection{Usage}
Pour mettre en oeuvre cette transformation, il faut utiliser la ligne de commande suivante :
\begin{center}
xsltproc -o \verb+<+nom du fichier d'arriv\'ee.html\verb+>+ MIF\_normalisedFormTester.xsl \verb+<+nom du fichier d'entr\'ee\verb+>+ 
\end{center}
(cf chapitre xsltproc pour plus d'informations)

\section{MIF\_fold.xsl}
\subsection{Pr\'esentation}
Cet outil permet \`a partir d'un fichier XML de format PSI d\'eroul\'e d'obtenir ce m\^eme fichier transform\'e dans le format PSI canonique en XML.
\subsection{Fonctionnalit\'e}
En entr\'ee il faut un fichier XML dans le format PSI d\'eroul\'e bien form\'e. Il existe l'outil MIF\_denormalisedFormTester.xsl qui permettra si on le souhaite de tester que le fichier d'entr\'ee est bien form\'e (cf partie suivante).\\
 \\
En sortie, nous obtenons un fichier XML dans le format PSI canonique. Ce fichier a les m\^emes informations que celles contenues dans le fichier de d\'epart si celui-ci \'etait bien form\'e.
\subsection{Usage}
Pour mettre en oeuvre cette transformation, il faut utiliser la ligne de commande suivante :
\begin{center}
xsltproc -o \verb+<+nom du fichier d'arriv\'ee.xml\verb+>+ MIF\verb+_+fold.xsl \verb+<+nom du fichier d'entr\'ee\verb+>+
\end{center}
(cf chapitre xsltproc pour plus d'informations)
\subsection{D\'epannage}
Si les conditions de d\'epart ne sont pas v\'erifi\'ees, c'est \`a dire que le fichier d'entr\'ee n'est pas bien form\'e, plusieurs cas de figure se pr\'esentent:
\begin{itemize} 
\item Il peut y a des r\'ef\'erences au lieu de descriptions dans les int\'eractions. Dans ce cas le r\'esultat restera correct.
\item Il peut y a des descriptions dans les listes (experimentList, interactorList et availabilityList),alors qu'en PSI d\'eroul\'e il ne doit rien y avoir dans les listes. Dans ce cas le r\'esultat restera correct mais ces descriptions seront perdues.
\item Il peut y avoir plusieurs descriptions avec le m\^eme attribut {\it id} mais ayant des contenus diff\'erents. Dans ce cas seul la premi\`ere descriptions rencontr\'ee avec un nouvel attribut sera conserv\'ee en PSI canonique.
\end{itemize}

\section{MIF\_denormalisedFormTester.xsl}
\subsection{Pr\'esentation}
Ce fichier est un test permettant de v\'erifier qu'un fichier PSI est bien PSI d\'eroul\'e.

\subsection{Fonctionnalit\'e}
En entr\'ee, il faut un fichier XML de format PSI.

En sortie, on a fichier HTML qui renvoie les erreurs (s'il y en a) que l'on a rencontr\'e et qui font que le fichier d'entr\'ee n'est pas en PSI d\'eroul\'e. Ces erreurs peuvent \^etre les suivantes :
\begin{itemize}
\item Celles appel\'ees {\it List of not allowed references :} qui indiquent lorsque l'on a une r\'ef\'erence dans les int\'eractions au lieu d'une description. \\Si ce champ est vide c'est qu'il n'y a aucune erreur. Par contre quand il y a une erreur, plusieurs informations sont indiqu\'ees :
\begin{itemize}
\item Le num\'ero de l'int�raction dans laquelle on a cette erreur.
\item L'attribut {\it ref} qui correspond \`a cette r\'ef\'erence.
\end{itemize}
\item Celles appel\'ees {\it List of misplaced definitions :} qui indiquent s'il existe des d\'efinitions dans les listes.\\Si ce champ est vide, c'est qu'il n'y a aucune erreur. Par contre, lorsqu'il y a une erreur, plusieurs informations sont indiqu\'ees :
\begin{itemize}
\item la position de l'experimentDescription dans la liste.
\item La valeur de l'attribut {\it id} correspondant \`a cette experimentDescription.
\end{itemize}
\item Celles appel\'ees {\it List of contradictory descriptions :} qui indiquent lorsque l'on a plusieurs descriptions ayant des attributs {\it id} identiques mais des contenus diff\'erents. \\Si ce champ est vide c'est qu'il n'y a aucune erreur. Par contre quand il y a une erreur, plusieurs informations sont indiqu\'ees :
\begin{itemize}
\item L'attribut {\it id} de la description dans laquelle on a trouv\'e une erreur.
\item Le contenu de la description mise en cause.
\item Le num\'ero de l'int\'eraction dans laquelle se trouve la description mise en cause.
\item Le contenu de la description qui sera gard\'ee lorsque l'on appliquera {\it MIF\verb+_+fold.xsl}.
\item Le num\'ero de l'int\'eraction dans laquelle se trouve la description qui sera retenue.
\end{itemize}
\end{itemize}
\subsection{Usage}
Pour mettre en oeuvre cette transformation, il faut utiliser la ligne de commande suivante :
\begin{center}
xsltproc -o \verb+<+nom du fichier d'arriv\'ee.html\verb+>+ MIF\_denormalisedFormTester.xsl \verb+<+nom du fichier d'entr\'ee\verb+>+
\end{center}
(cf chapitre xsltproc pour plus d'informations)

\section{MIF\_view.xsl}
\subsection{Pr\'esentation}
Cet outil permet \`a partir d'un fichier XML PSI sous forme canonique d'obtenir un fichier HTML qui pr�sente les int�raction sous forme synth�tique.
\subsection{Fonctionnalit\'e}
En entr\'ee il faut un fichier XML dans le format PSI canonique bien form\'e et conforme � la XSD. Il existe l'outil MIF\_normalisedFormTester.xsl qui permettra si on le souhaite de tester que le fichier d'entr\'ee est bien form\'e (cf partie suivante).\\
\\
En sortie, nous obtenons un fichier HTML qui pr�sente les int�raction. On affiche les sources et les licences du fichiers. Puis on construit un tableau dont chaque ligne est un participant � une int�raction. Le tableau pr�sente les donn�es : {\em interactorRef/@ref} or {\em proteinInteractor/names/shortLabel}, {\em participant/role}, {\em participant/isTaggedProtein}, {\em participant/isOverexpressedProtein}, {\em interactionType}. Pour plus de lisibilit� les lignes sont regroup�es par int�raction et on alterne la couleur d'affichage d'un partiant � chaque linge.

\subsection{Usage}
Pour mettre en oeuvre cette transformation, il faut utiliser la ligne de commande suivante :
\begin{center}
xsltproc -o \verb+<+nom du fichier d'arriv\'ee.html\verb+>+ MIF\verb+_+view.xsl \verb+<+nom du fichier d'entr\'ee.xml\verb+>+
\end{center}
(cf chapitre xsltproc pour plus d'informations)

\subsection{D\'epannage}
Si les conditions de d\'epart ne sont pas v\'erifi\'ees, c'est \`a dire que le fichier d'entr\'ee n'est pas bien form\'e, les donn�es ne seront pas affich�s correctements.


